\documentclass[12pt]{article}
\usepackage[margin=1in]{geometry}
\usepackage{amsmath}
\usepackage{booktabs}


\begin{document}
CS6041 HW2, Noah Gardner, 000843905\newline

\section{Assignment 2}

\begin{enumerate}
    \item You are given sample documents with corresponding classes. Documents
          are annotated as A, or B. These documents are splitted as training and
          testing set shown in the below table. Assign the most probable class
          to the test sentence given below using naïve bayes classification
          approach. Please mention each step clearly (i.e., prior probability,
          conditional probability, etc.). \textbf{[Points 25]}
          \begin{equation}
              P(c|X) = \frac{P(X|c)P(c)}{P(X)}
          \end{equation}
          \begin{figure}[ht]
              \centering
              \begin{tabular}{|l|l|}
                  \hline\hline
                  $c (prior)$        & $P(c)$                              \\
                  \hline
                  $A$                & $P(A) = 8/11 = 0.72$                \\
                  $B$                & $P(B) = 3/11 = 0.27$                \\
                  \hline\hline
                  $x (predictor)$    & $P(X)$                              \\
                  \hline
                  $Chinese$          & $P(Chinese) = 6/11 = 0.55$          \\
                  $Japan$            & $P(Japan) = 1/11 = 0.27$            \\
                  $Tokyo$            & $P(Tokyo) = 1/11 = 0.27$            \\
                  \hline\hline
                  $X|c (likelihood)$ & $P(X|c)$                            \\
                  \hline
                  $Chinese|A$        & $P(Chinese|A) = 1/3 = 0.33$         \\
                  $Japan|A$          & $P(Japan|A) = 1/3 = 0.33$           \\
                  $Tokyo|A$          & $P(Tokyo|A) = 1/3 = 0.33$           \\
                  \hline
                  $Chinese|B$        & $P(Chinese|B) = 5/8 = 0.625$        \\
                  $Japan|B$          & $P(Japan|B) = 0/8 = 0.0$            \\
                  $Tokyo|B$          & $P(Tokyo|B) = 0/8 = 0.0$            \\
                  \hline\hline
                  $probability$      & $P(c|X)$                            \\
                  \hline
                  $A|Chinese$        & $(0.33) * (0.72) / (0.55) = 0.432$  \\
                  $A|Japan$          & $(0.33) * (0.72) / (0.27) = 0.88$   \\
                  $A|Tokyo$          & $(0.33) * (0.72) / (0.27) = 0.88$   \\
                  \hline
                  $B|Chinese$        & $(0.625) * (0.27) / (0.55) = 0.307$ \\
                  $B|Japan$          & $(0.0) * (0.27) / (0.27) = 0.0$     \\
                  $B|Tokyo$          & $(0.0) * (0.27) / (0.27) = 0.0$     \\
                  \hline\hline
              \end{tabular}
              \caption{Naive Bayes Classification calculations based on input training set.}
          \end{figure}
          \begin{description}
              \item[Answer] Since there are no instances of $Japan$ or $Tokyo$
                  in the training set for class $B$, the output of the model
                  will necessarily be $0\%$ for class $B$. Therefore, the class
                  of the input text will be classified as $A$.

                  There exist some methods that can handle the problem of $0\%$
                  likelihood for Naive Bayes classifiers. One such method is to
                  ignore any words that did not appear in the training set. If
                  this method is applied to the input text, the probability of
                  each class is calculated as follows:

                  \begin{equation}
                      P(A|'test') = P(A|Chinese)^3 * P(A|Japan) * P(A|Tokyo)
                  \end{equation}
                  \begin{equation*}
                      = 0.432^3 * 0.88 * 0.88 = 0.0624
                  \end{equation*}

                  \begin{equation}
                      P(B|'test') = P(B|Chinese)^3
                  \end{equation}
                  \begin{equation*}
                      = 0.307^3 = 0.0289
                  \end{equation*}

                  The probability that the test text is higher in class $A$ than
                  in class $B$. Therefore, the class of the input text will be
                  classified as $A$.
          \end{description}
    \item What is cross-validation? Would you please give an example and explain
          how cross-validation work? When to use cross-validation during an
          experiment? \textbf{[Points 15]}
          \begin{description}
              \item[Answer] Cross-validation is a technique that splits the data
                  into different sections (or \textit{folds}) and tests the
                  algorithm on each fold. This method allows one to prevent
                  overfitting by using different data for each training set.
                  Then, after training the algorithm once on each fold, the
                  models are compared to see which one performs better or can
                  also be average to evaluate the performance of the algorithm.

                  An example of cross-validation is \textit{k-fold
                      cross-validation}. For example, if examine an algorithm
                  using $2$-fold cross-validation, we would split the
                  dataset into the training and testing sets. Then, we would
                  split the training set into two folds. We would train the
                  algorihtm twice (once on each fold) and use the fold that
                  wasn't used for training for testing or validation.
          \end{description}
    \item Explain how gradient descent algorithm works? Please explain the
          effect of learning rate on the learning algorithm while updating
          parameters using the following equations. Please show differences
          among different types of gradient descent – mini-batch, batch, and
          stochastic gradient descent. \textbf{[Points 15]}
          \begin{equation}
              \textbf{w}^{(t+1)} = \textbf{w}^{(t)} - \eta\frac{d}{dw}f(x,\textbf{w})
          \end{equation}

          \begin{description}
              \item[Answer] The goal of the gradient descent algorithm is to
                  find a valley in the function. It is often compared to
                  climbing down a mountain. If one was climbing down a mountain,
                  they would not want to climb past the lowest point. Instead,
                  their goal would be to move as much as possible towards the
                  bottom (to reach the bottom as soon as possible), without
                  overshooting (which would require the climber to climb back
                  down).

                  The 'rate' at which our climber climbs down the mountain is
                  similar to the learning rate ($\eta$) in the gradient descent
                  algorithm. We can increase the learning rate to make it to the
                  bottom of the expression (\textit{minima}, a possible solution
                  for the equation). $ \textbf{w}^{(t)}$ represents the current
                  value of the parameter and $ \textbf{w}^{(t+1)}$ represents
                  the new value of the parameter after the next update step. The
                  parameter $\frac{d}{dw}f(x,\textbf{w})$ is the derivative of
                  the function $f(x,\textbf{w})$ with respect to the parameter
                  $w$, which gives us the slope of loss at the current point.

                  There are different methods for applying the gradient descent
                  algorithm, including mini-batch, batch, and stochastic
                  gradient descent. Sotchastic gradient descent is an iterative
                  method which updates the gradient at each step. The batch
                  gradient descent is a method that updates the gradient at the
                  end of each epoch by using a \textit{batch} of the data
                  (usually the whole dataset). Mini-batch gradient descent is a
                  method that is derived from batch gradient descent and uses
                  smaller batches of data to update the gradient in fewer steps
                  than stochastic gradient descent but more steps than batch
                  gradient descent.
          \end{description}

    \item You are given the following text documents. Please answer the
          following questions. \textbf{[Points 45]}
          \begin{description}
              \item \textbf{Text:}
              \item It is going to rain today.
              \item Today I am not going outside.
              \item NLP is an interesting topic.
              \item NLP includes ML, DL topics too.
              \item I am going to complete NLP homework, today.
          \end{description}

          \begin{enumerate}
              \item Would you please calculate the TF-IDF vector for each of the
                    tokens? Please show each step (i.e., tokenization,
                    vocabulary, TF, IDF, etc.). \textbf{[Points 15]}
                    \begin{description}
                        \item Answer:
                        \item[Step 1:] Split into tokens, remove punctuation,
                            and set tokens to lower-case.
                        \item \verb/[['it', 'is', 'going', 'to', 'rain', 'today'],/
                        \item \verb/['today', 'i', 'am', 'not', 'going', 'outside'],/
                        \item \verb/['nlp', 'is', 'an', 'interesting', 'topic'],/
                        \item \verb/['nlp', 'includes', 'ml', 'dl', 'topics', 'too'],/
                        \item \verb/['i', 'am', 'not', 'going', 'to', 'play', 'football', 'today']]/
                        \item[Step 2:] Create a vocabulary from the tokens (I
                            use a sorted vocabulary here).
                        \item \verb/['am', 'an', 'dl', 'football', 'going', 'i', /
                        \item \verb/'includes', 'interesting', 'is', 'it', 'ml',/
                        \item \verb/'nlp', 'not', 'outside', 'play', 'rain', 'to',/
                        \item \verb/'today', 'too', 'topic', 'topics']/
                        \item[Step 3:] Create a TF vector for each token.
                    \end{description}

                    \begin{figure}
                        \begin{tabular}{llllllrlllll}
\toprule
{} & tf\_d0 & tf\_d1 & tf\_d2 & tf\_d3 & tf\_d4 &    idf & tf\_idf\_d0 & tf\_idf\_d1 & tf\_idf\_d2 & tf\_idf\_d3 & tf\_idf\_d4 \\
\midrule
am          &     0 &     1 &     0 &     0 &     1 &  0.398 &       0.0 &     0.398 &       0.0 &       0.0 &     0.398 \\
an          &     0 &     0 &     1 &     0 &     0 &  0.699 &       0.0 &       0.0 &     0.699 &       0.0 &       0.0 \\
dl          &     0 &     0 &     0 &     1 &     0 &  0.699 &       0.0 &       0.0 &       0.0 &     0.699 &       0.0 \\
football    &     0 &     0 &     0 &     0 &     1 &  0.699 &       0.0 &       0.0 &       0.0 &       0.0 &     0.699 \\
going       &     1 &     1 &     0 &     0 &     1 &  0.222 &     0.222 &     0.222 &       0.0 &       0.0 &     0.222 \\
i           &     0 &     1 &     0 &     0 &     1 &  0.398 &       0.0 &     0.398 &       0.0 &       0.0 &     0.398 \\
includes    &     0 &     0 &     0 &     1 &     0 &  0.699 &       0.0 &       0.0 &       0.0 &     0.699 &       0.0 \\
interesting &     0 &     0 &     1 &     0 &     0 &  0.699 &       0.0 &       0.0 &     0.699 &       0.0 &       0.0 \\
is          &     1 &     0 &     1 &     0 &     0 &  0.398 &     0.398 &       0.0 &     0.398 &       0.0 &       0.0 \\
it          &     1 &     0 &     0 &     0 &     0 &  0.699 &     0.699 &       0.0 &       0.0 &       0.0 &       0.0 \\
ml          &     0 &     0 &     0 &     1 &     0 &  0.699 &       0.0 &       0.0 &       0.0 &     0.699 &       0.0 \\
nlp         &     0 &     0 &     1 &     1 &     0 &  0.398 &       0.0 &       0.0 &     0.398 &     0.398 &       0.0 \\
not         &     0 &     1 &     0 &     0 &     1 &  0.398 &       0.0 &     0.398 &       0.0 &       0.0 &     0.398 \\
outside     &     0 &     1 &     0 &     0 &     0 &  0.699 &       0.0 &     0.699 &       0.0 &       0.0 &       0.0 \\
play        &     0 &     0 &     0 &     0 &     1 &  0.699 &       0.0 &       0.0 &       0.0 &       0.0 &     0.699 \\
rain        &     1 &     0 &     0 &     0 &     0 &  0.699 &     0.699 &       0.0 &       0.0 &       0.0 &       0.0 \\
to          &     1 &     0 &     0 &     0 &     1 &  0.398 &     0.398 &       0.0 &       0.0 &       0.0 &     0.398 \\
today       &     1 &     1 &     0 &     0 &     1 &  0.222 &     0.222 &     0.222 &       0.0 &       0.0 &     0.222 \\
too         &     0 &     0 &     0 &     1 &     0 &  0.699 &       0.0 &       0.0 &       0.0 &     0.699 &       0.0 \\
topic       &     0 &     0 &     1 &     0 &     0 &  0.699 &       0.0 &       0.0 &     0.699 &       0.0 &       0.0 \\
topics      &     0 &     0 &     0 &     1 &     0 &  0.699 &       0.0 &       0.0 &       0.0 &     0.699 &       0.0 \\
\bottomrule
\end{tabular}

                    \end{figure}

              \item  Please calculate term-term co-occurrence matrix with a
                    given context window ±3. \textbf{[Points 15]}
              \item Please find the most similar words ( a pair of words) from
                    their vector representations for both TF-IDF and
                    co-occurrence matrix based vector representation cases. To
                    compute most similarity score, you may use cosine similarity
                    score. Show details similarity calculation for both cases –
                    (a) TF-IDF and (b) co-occurrences-based representations.
                    \textbf{[Points 15]}
          \end{enumerate}





\end{enumerate}
\section{Appendix}
\end{document}
