\documentclass[12pt]{article}
\usepackage[margin=1in]{geometry}
\usepackage{amsmath}
% \usepackage{showframe}

\begin{document}
CS6041 HW1, Noah Gardner, 000843905\newline
Your assignment must be typed in Word or Latex for exact ONE question per page,
and turn in PDF files. Non-typed submissions will NOT be graded.
\newline\newline
\textbf{Problem 1. }\textit{Find the expression that is the contrapositive of $A
        \lor \neg B \to C \lor \neg D$.} \newline

Apply negation of both sides.
\begin{equation}
    \neg (A \lor \neg B) \to \neg (C \lor \neg D)
\end{equation}

\begin{equation}
    \neg A \wedge B \to \neg C \wedge D
\end{equation}

Reversal of hypothesis and conclusion.
\begin{equation}
    \neg C \wedge D \to \neg A \wedge B
    \label{eq:3}
\end{equation}

\eqref{eq:3} is the contrapositive of the expression $A \lor \neg B \to C \lor
    \neg D$.

\newpage
\textbf{Problem 2. }\textit{Prove the statement $S(n) = \sum_{i=2}^n i =
        (n+2)(n-1)/2$ by induction.} \newline

\textbf{Basis. } Show that the statement is true for $n=2$.
\begin{equation}
    S(2) = (2+2)(2-1)/2 = 2
\end{equation}

\textbf{Hypothesis. } Basis $n=2$ is true. Assume the statement is true when $n$
is some value larger than 2, such as $n=k$. That is, \newline
\begin{equation}
    S(k) = \sum_{i=2}^k i = (k+2)(k-1)/2
    \label{eq:5}
\end{equation}

\textbf{Induction. } Prove that the statement holds when $n=k+1$, given $k>=2$.
\begin{equation}
    S(k+1) = \sum_{i=2}^{k+1} i = k(k+3)/2
    \label{eq:6}
\end{equation}

Rule of summation.
\begin{equation}
    \sum_{i=2}^{k+1} i = \sum_{i=2}^k i + (k+1)
    \label{eq:7}
\end{equation}

Apply hypothesis \eqref{eq:5} to \eqref{eq:7}.
\begin{equation}
    \sum_{i=2}^k i + (k+1) = (k+2)(k-1)/2 + (k+1)
\end{equation}

Factor out multiplication.
\begin{equation}
    (k+2)(k-1)/2 \to (k^2+1k-2)/2
    \label{eq:9}
\end{equation}

Apply common denominator of 2.
\begin{equation}
    (k+1) \to (2k+2)/2
    \label{eq:10}
\end{equation}

Add the two components from \eqref{eq:9} and \eqref{eq:10}.
\begin{equation}
    (k^2+1k-2)/2 + (2k+2)/2 = (k^2+3k)/2
\end{equation}

Factor out $k$.
\begin{equation}
    (k^2+3k)/2 = k(k+3)/2
    \label{eq:12}
\end{equation}

\eqref{eq:12} is equivalent to \eqref{eq:6}, so the statement $S(n) =
    \sum_{i=2}^n i = (n+2)(n-1)/2$ is proven by induction.

\newpage
\textbf{Problem 3. }\textit{Prove the statement $n^n/3^n < n!$ for $n \geq 6$ by
    induction.}

\textbf{Basis. } Show that the statement is true for $n=6$.
\begin{equation}
    S(6) = 6^6/3^6 < 6! = 2^6 < 720 = 64 < 720
\end{equation}

\textbf{Hypothesis. } Basis $n=6$ is true. Assume the statement is true when $n$
is some value larger than 6, such as $n=k$. That is, \newline
\begin{equation}
    S(k) = k^k/3^k < k!
    \label{eq:14}
\end{equation}

If \eqref{eq:14} is true, then it also follows that:
\begin{equation}
    k^k < 3^k*k!
    \label{eq:15}
\end{equation}

\textbf{Induction. } Prove that the statement holds when $n=k+1$, given $k \geq
    6$.
\begin{equation}
    S(k+1) = (k+1)^{(k+1)}/3^{(k+1)} < (k+1)!
\end{equation}

Multiply both sides by $3^{(k+1)}$, factor out $3$.
\begin{equation}
    (k+1)^{(k+1)} < 3*3^k*(k+1)!
\end{equation}

Divide both sides by $k+1$.
\begin{equation}
    (k+1)^{k} < 3*3^k*k!
\end{equation}

Apply hypothesis \eqref{eq:15}.
\begin{equation}
    (k+1)^{k} < 3*k^{k}
\end{equation}

Raise both sides to power to $1/k$
\begin{equation}
    (k+1) < 3k
    \label{eq:20}
\end{equation}

\eqref{eq:20} holds true, so the statement $S(n) = n^n/3^n < n!$ for $n \geq 6$
is proven by induction.

\end{document}
