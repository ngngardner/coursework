\documentclass[12pt]{article}
\usepackage[margin=1in]{geometry}
\usepackage{amsmath}
\usepackage{booktabs}
\usepackage{esvect}

\begin{document}
CS7357 HW1, Noah Gardner, 000843905\newline

\section{Assignment 1}
\begin{enumerate}
    \item \textbf{[Points 30]} We have given 5 students aptitude test marks and their
          statistically computed grades below.

          \begin{table}[h]
              \centering
              \begin{tabular}{|l|l|l|}
                  \hline
                  Student & Marks & Grade \\
                  \hline
                  1       & 95    & 85    \\
                  \hline
                  2       & 85    & 95    \\
                  \hline
                  3       & 80    & 70    \\
                  \hline
                  4       & 70    & 65    \\
                  \hline
                  5       & 60    & 70    \\
                  \hline
              \end{tabular}
          \end{table}
          \begin{enumerate}
              \item \textbf{[Points 20]} We are interested to compute grades from
                    given test marks using linear regression. Can you please
                    estimate the line equation using the normal equation method?
                    Please show the detailed computation of each step.  Also, show
                    the final line equation and its parameters.

                    Suppose $h(\theta) = \theta_0 + \theta_1 x_1$

                    $\vv{h(\theta^*)} = \begin{bmatrix}
                            85 \\
                            95 \\
                            70 \\
                            65 \\
                            70
                        \end{bmatrix} $

                    $ X = A = \begin{bmatrix}
                            1 & 95 \\
                            1 & 85 \\
                            1 & 80 \\
                            1 & 70 \\
                            1 & 60
                        \end{bmatrix} $

                    $A^T = \begin{bmatrix}
                            1  & 1  & 1  & 1  & 1  \\
                            95 & 85 & 80 & 70 & 60
                        \end{bmatrix} $

                    $A^T*A = \begin{bmatrix}
                            a_1 & a_2 \\%
                            a_3 & a_4
                        \end{bmatrix}$
                    $ = \begin{bmatrix}
                            5   & 390   \\
                            390 & 31150 \\
                        \end{bmatrix}$

                    where $a_1 = 1*1 + 1*1 + 1*1 + 1*1 + 1*1 = 5$

                    $a_2 = 1*95 + 1*85 + 1*80 + 1*70 + 1*60 = 390$

                    $a_3 = 95*1 + 85*1 + 80*1 + 70*1 + 60*1 = 390$

                    $a_4 = 95*95 + 85*85 + 80*80 + 70*70 + 60*60 = 31150$

                    Finally,

                    $ (A^T A)^{-1} = \begin{bmatrix}
                            8.53425   & -0.106849  \\
                            -0.106849 & 0.00136986 \\
                        \end{bmatrix}$

                    $ \vv{\theta^*} = (A^T A)^{-1}*A^T*\vv{h(\theta^*)} = \begin{bmatrix}
                            26.7808  \\
                            0.643836 \\
                        \end{bmatrix}$

                    Therefore, by the normal equation, the line going through
                    the sample data is:
                    $h(\theta) = 26.7808 + 0.643836 x_1$

              \item \textbf{[Points 10]} Predict grades for the following given student marks using your
                    computed equation in step 1.
                    Given marks \verb/[65, 75, 77, 83, 87]/

                    The output values are \verb/[68.6301, 75.0685, 76.3562, 80.2192, 82.7945]/
          \end{enumerate}
\end{enumerate}

\end{document}