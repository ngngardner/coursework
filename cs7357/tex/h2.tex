\documentclass[12pt]{article}
\usepackage[margin=1in]{geometry}
\usepackage{amsmath}
\usepackage{booktabs}
\usepackage{esvect}
\usepackage{float}

\begin{document}
CS7357 HW2, Noah Gardner, 000843905\newline

\section{Assignment 1}
\begin{enumerate}
    \item \textbf{[Points 10]} We learned two linear model - linear regression
          and logistic regression. Compare both methods. Can we use linear
          regression model to detect person face in an image? Describe your
          rationale behind it.

          \textbf{Answer:}
          Linear regression is a linear model that is used to model a linear
          relationship - as an input variable is related to an output variable.
          Logisitic regressions is a linear model that is used to model a binary
          output - for example, given an input variable, the output variable is
          either true or false.

          For detected a face in an image, it might not make sense to use a
          linear model, unless we are trying to count the number of faces in the
          image. Instead, we can use a logistic regression model to output
          'contains a face' or 'does not contain a face' based on the input
          image.

    \item \textbf{[Points 15]} In logistic regression classifier, we are fitting
          a s-shape curve to fit the data. We are given 10 sample points with
          corresponding probabilities as follows, 0.34, 0.21, 0.54, 0.45, 0.60,
          0.70, 0.80, 0.95, 0.99.

          \begin{enumerate}
              \item \textbf{[Points 7]} What is log odds? Compute log odds
                    values for those given data points.

                    \textbf{Answer:}
                    Odds is a ratio of the probability of success to the
                    probability of failure. Log odds is the logarithm of the
                    odds.

                    \begin{table}[h]
                        \centering
                        \begin{tabular}{l|l|l}
                            \textbf{Probability} & \textbf{Odds} & \textbf{Log Odds} \\
                            \hline
                            0.34                 & 0.52          & -0.65             \\
                            0.21                 & 0.27          & -1.31             \\
                            0.54                 & 1.17          & 0.16              \\
                            0.45                 & 0.82          & -0.2              \\
                            0.6                  & 1.5           & 0.41              \\
                            0.7                  & 2.33          & 0.85              \\
                            0.8                  & 4.0           & 1.39              \\
                            0.95                 & 19.0          & 2.94              \\
                            0.99                 & 99.0          & 4.6               \\
                        \end{tabular}
                    \end{table}
              \item \textbf{[Points 8]} Compute log likelihood for this given
                    data points.
          \end{enumerate}

    \item \textbf{[Points 15]} We know logistic regression is a binary
          classifier. Can we use it for multiclass classification? Provide
          detail rationale behind your answer and include any drawback of your
          proposed approaches.

          \textbf{Answer:}
          Since logistic regression is a binary classifier, we can use it for
          multiclass classification with some considerations. First, we might
          consider a series of models which classifies the input as 'A' or 'Not
          A'. If the input is classified as 'Not A', then we can move to the
          next model, which classifies the input as 'B' or 'Not B'. A problem
          with this kind of approach is that the accuracy can be low.

          Another approach is to compare each class to each other class. We
          might have models such as 'A' or 'B', and 'A' or 'C'. A problem with
          this kind of approach is that it can be slow.

    \item \textbf{[Points 10]} SoftMax is a multiclass classifier, and it
          converts logits to probabilities. We are given logit values 3.5, 6.1,
          -2.9, -1.2 for 4 classes “bus”, “truck”, “car”, “van”, respectively.
          Compute the probability of those given logits and classify it.

    \item \textbf{[Points 25]} We are designing a 2-layer feedforward neural
          network. Our input features are 3-dimensional. The first hidden layer
          has 5 neurons with sigmoid activation function. Final layer contains
          two neurons with Relu activation function. Assume that given inputs
          are $x_1, x_2, x_3$ and hidden layer weights are $w_{ij}^{l}$, where
          $l \in {1, 2}$ is the layer number and $i \in {1,2,3}$ is the number
          of inputs. $j \in {1,2,3,4,5}$ indicates the number of neurons.
          $b_{j}^{j}$ indicates bias for corresponding layers and neurons. For
          any inconsistency with the notation given, you can modify it and
          mentioned the notation scheme in your answer.

          \begin{enumerate}
              \item \textbf{[Points 10]} Draw a complete diagram of this feed
                    forward neural network showing all individual weights,
                    biases.
              \item \textbf{[Points 10]} Show forward computation for this given
                    input $x_1, x_2, x_3$. Show detailed equations for each
                    computing unit (neuron) for each layer.
          \end{enumerate}
\end{enumerate}
\end{document}